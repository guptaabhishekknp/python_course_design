\documentclass{article}
\usepackage[margin=0.7in]{geometry}
\usepackage{tikz}
\usetikzlibrary{arrows.meta}

\usepackage{amsmath}
\usepackage{enumitem}
\usepackage{tikz}
\usepackage{mwe}
\usepackage{float}
\usepackage{listings}
\usepackage{color}
% Tikz settings optimized for causal graphs.
% Just copy-paste this part
\usetikzlibrary{shapes,decorations,arrows,calc,arrows.meta,fit,positioning}
\tikzset{
    -Latex,auto,node distance =1 cm and 1 cm,semithick,
    state/.style ={circle, draw, minimum width = 0.7 cm},
    point/.style = {circle, draw, inner sep=0.04cm,fill,node contents={}},
    bidirected/.style={Latex-Latex,dashed},
    el/.style = {inner sep=2pt, align=left, sloped},
    ->,>=stealth'
}

\definecolor{backcolour}{rgb}{0.94,0.97,1.0}

\lstdefinestyle{customc}{
  backgroundcolor=\color{backcolour},
  belowcaptionskip=1\baselineskip,
  breaklines=true,
  %frame=L,
  xleftmargin=\parindent,
  language=python,
  showstringspaces=false,
  basicstyle=\footnotesize\ttfamily,
  keywordstyle=\bfseries\color{green!40!black},
  commentstyle=\itshape\color{purple!40!black},
  identifierstyle=\color{blue},
  stringstyle=\color{orange},
}

\lstdefinestyle{customasm}{
  belowcaptionskip=1\baselineskip,
  frame=L,
  xleftmargin=\parindent,
  language=[x86masm]Assembler,
  basicstyle=\footnotesize\ttfamily,
  commentstyle=\itshape\color{purple!40!black},
}

\lstset{escapechar=@,style=customc}

\usepackage[utf8]{inputenc}
\title{Week 11 - interpreting error messages, debugging}
\author{Math16b}
\date{}
\begin{document}
\maketitle


\section{List of Common Errors}
\begin{enumerate}
    \item    \begin{lstlisting} 
NameError: name 'some_name' is not defined
NameError: global name 'some_name' is not defined
    \end{lstlisting}
    This error occurs when you use a variable name which has not been declared before.
    
    \item \begin{lstlisting} 
IndentationError: unindent does not match any outer indentation level
IndentationError: unexpected indent
IndentationError: expected an indented block

    \end{lstlisting}
     In order to indent the code correctly, one must consistently use one and only one of either: tab or: four spaces. These errors occur if there are tabs mixed with four spaces. These can also occur if there is inconsistent space-indenting within a single block.
     
     \item \begin{lstlisting}
SyntaxError: invalid syntax
     \end{lstlisting}
     These can happen in a variety of sitations. Some of them are:
     \begin{itemize}
         \item Forgetting the parens around the arguments to print
         \item Forgetting the colon at the end of the condition in an if statement
         \item Trying to use a reserved word as a variable name

     \end{itemize}
     
    \item \begin{lstlisting}
TypeError: unsupported operand type(s) for +: 'int' and 'str'

    \end{lstlisting}
    A type error can occur when you do an operation on a data type which is not correct. For example, in the above shown error, the problem arises because int+string does not make sense.
    
    \item \begin{lstlisting}
AttributeError: 'module' object has no attribute 'sparse'
    \end{lstlisting}
    An attribute in Python means some property that is associated with a particular type of object. In other words, the attributes of a given object are the variables and functions it has in it. Attribute errors in Python are generally raised when you try to access or call an attribute that a particular object type doesn’t possess. 
    
    \item \begin{lstlisting}
ValueError: could not convert string to float: 'string'

    \end{lstlisting}
    A Value error is raised when a built-in operation or function receives an argument that has the right type but an inappropriate value. In the above example, the float function can take a string, ie float('5'). The error arises when the value 'string' in float('string') is an inappropriate (non-convertible) string. \\
    Or for example when we try to add two numpy arrays of different sizes.
    
    \item \begin{lstlisting}
KeyError: 'name'
    \end{lstlisting}
    These errors occur when dealing with python dictionaries. A keyError raised when a dictionary key is not found in the set of existing keys.
    
    \item \begin{lstlisting}
IndexError: list index out of range
    \end{lstlisting}
    This error occurs when we are trying to refer to some index that doesn't exist.
    
    \item \begin{lstlisting}
ModuleNotFoundError
    \end{lstlisting}
    This occurs when a module is not found.

\item \begin{lstlisting}
ImportError
\end{lstlisting}
This occurs when a specified function can not be found.

\item \begin{lstlisting}
ZeroDivisionError: division by zero
\end{lstlisting}
This occurs when you divide by zero.
\end{enumerate}
\section{Exercises}
The code snippets below raise one of the above errors. For each code snippet, indicate the error it raises.

\begin{enumerate}
    \item \begin{lstlisting}
for i in range(10)
    print(i**2)
    \end{lstlisting}
    
    \item \begin{lstlisting}
import numpy as np
A = numpy.array([1,2,3])
    \end{lstlisting}
    
    \item \begin{lstlisting}
a = 1
print(A)
    \end{lstlisting}
    
\item \begin{lstlisting}
l = ['a',2, True]
print(l(0))
\end{lstlisting}

\item \begin{lstlisting}
ls = [1,2,3,4]
for i in ls:
    print(ls[i]+ls[i+1])
\end{lstlisting}

\item \begin{lstlisting}
print(2+'2')
\end{lstlisting}


\item \begin{lstlisting}
l = 1
l.append(3)
\end{lstlisting}
\end{enumerate}
\end{document}